\usepackage[a2paper, margin=0cm]{geometry}
\usepackage{amsmath}
\usepackage{nicefrac}
\usepackage{upgreek}
\usepackage{xcolor}
\definecolor{link}{HTML}{4b69a4}
\renewcommand{\eqref}[1]{\textcolor{link}{\textbf{(\ref{#1})}}}

\usepackage{tikz}
\usetikzlibrary{matrix,backgrounds,decorations.pathreplacing,calc,shapes.misc,intersections}

\usepackage[makeroom]{cancel}
\newcommand\Ccancel[2][black]{
    \let\OldcancelColor\CancelColor
    \renewcommand\CancelColor{\color{#1}}
    \cancel{#2}
    \renewcommand\CancelColor{\OldcancelColor}
}

\newcommand{\commentOut}[1]{}

\newcommand{\highlight}[2][yellow]{\colorbox{#1}{$#2$}}

\newcommand{\bhighlight}[3][O{blue!40}]{%
  \draw[rounded corners,line width=1bp,color=#1] (#2.north west) rectangle (#3.south east);
}

\newcommand{\fhighlight}[3][O{blue!40}]{%
  \fill[rounded corners,fill=#1] (#2.north west) rectangle (#3.south east);
}

\def\centerarc[#1](#2)(#3:#4:#5)% Syntax: [draw options] (center) (initial angle:final angle:radius)
    { \draw[#1] ($(#2)+({#5*cos(#3)},{#5*sin(#3)})$) arc (#3:#4:#5); }
\tikzset{partial ellipse/.style args={#1:#2:#3}{insert path={+ (#1:#3) arc (#1:#2:#3)}}}
\tikzset{cross/.style={cross out, draw=black, minimum size=2*(#1-\pgflinewidth), inner sep=0pt, outer sep=0pt},
         cross/.default={1pt}}%default radius will be 1pt.
\tikzset{ext/.pic={\path [fill=white] (-0.2,0)to[bend left](0,0.1)to[bend right](0.2,0.2)to(0.2,0)to[bend left](0,-0.1)to[bend right](-0.2,-0.2)--cycle;
                   \draw (-0.2,0)to[bend left](0,0.1)to[bend right](0.2,0.2) (0.2,0)to[bend left](0,-0.1)to[bend right](-0.2,-0.2);}}

\newdimen\XCoord
\newdimen\YCoord
\newcommand*{\ExtractCoordinate}[1]{\path (#1); \pgfgetlastxy{\XCoord}{\YCoord};}%

% Somehow using /dot{...} can break doxygen's alias system. So the next equation using \fl will not be rendered.
% This however can be fixed by surrounding the \dot by a \mathbf, \mathnormal, ...
\newcommand{\dotup}[1]{\mathnormal{\dot{#1}}}
\newcommand{\ddotup}[1]{\mathnormal{\ddot{#1}}}